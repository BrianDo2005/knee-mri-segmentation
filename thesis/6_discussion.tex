\section{Discussion}

In this thesis I presented the development of a fully automated workflow based on convolutional neural networks, that segments bones in three dimensional MRI data of human knees. It shows excellent performance of 98\% DSC while distinguishing between Femur, Tibia and Fibula. The network shows surprisingly accurate results on sagittal images although it was entirely trained using coronal data. It is also applicable to changes in crop and resolution due to its fully convolutional structure. Furthermore, I was able to combine the masked bone images with part of the architecture as a pre-trained model to predict the age of the candidates with a MAE of less than 200 days. All of this is possible using non-invasive magnetic resonance imaging.

The segmentation results show improved performance when compared to other studies and further prove the importance of neural networks in the medical field. Although processing speed was no priority, the architecture consists of only 210,000 parameters making it 150 times smaller than U-Net. On a sub-\$1000 workstation it can be trained from scratch in 1 to 4 hours depending if the bones need to be separated. Using the pre-trained weights, one can likely apply transfer learning with very little effort to solve similar problems.

The robustness against different resolutions begs the question if training can be sped up by first training a model on lower resolution images before using the original data.


what they did (briefly)

what they found
what were the significant, memorable findings?
what do the findings mean?
what does it mean that X was rated as 4.61 and Y was rated as 3.93?
do the best of your knowledge, why do you think that is? what accounts for these results?
why are the findings significant/important/useful? how can they be used, and who can use them?

\newpage
\section{Background}

\subsection{Medicine}

write something

\subsubsection{Magnetic Resonance Imaging}

text

\subsubsection{Epiphyseal Plates}

text

\subsection{Computer Science}

\subsubsection{Artificial Intelligence*}

Artificial Intelligence (AI) is understood as the effort of automating a given task that normally needs human intelligence to solve\cite{Chollet2017}. The history of AI goes back to the 1950s, where a certain type called "symbolic AI" started to gain popularity. It was believed that human level intelligence could be achieved through hard-coded rules that programmers specified. 

Taking a complex problem like playing chess and continuing to break it into smaller problems, until they can be solved with known logic. While it was effective for certain tasks, fuzzy problems like image classification, speech recognition or language translation were difficult to tackle. Over the years a new approach was found, which today is referred to as machine learning.

\subsubsection{Machine Learning*}

The concept of classical programming is that an engineer defines a set of rules, called an algorithm, which uses input data to calculate some form of output data\cite{Chollet2017}.

--insert picture--

A machine learning algorithm is an algorithm that is able to learn from data\cite{Goodfellow2016}. It can be used to automatically calculate these rules, so they don't have to be specified by hand. Three components are needed for such an approach.

\begin{itemize}
\item Input data the algorithm is supposed to transform
\item Output data the algorithm is supposed to predict
\item A measurement to validate the performance of a prediction
\end{itemize}

It works by feeding input and output data into a pipeline, which will learn to transform one into the other. With the advantage that no explicit programming is needed to generate the rules, comes the disadvantage that prior input and output data is needed for the initial learning process.

--insert picture--

ML is an effective method if it's not feasible or possible to define an algorithm by hand and sufficient data is available for training. How much “sufficient” is depends on factors like type of task, complexity of the data, uniformity of the data, type of ML algorithm and others.

There are different subparts to machine learning like supervised and unsupervised learning. Supervised learning is used when it's clear what the output data looks like, whereas unsupervised learning can help to find unknown patterns in the data. Examples of supervised learning techniques include linear regression, naive bayes, support vector machines, decision trees, random forests, gradient boosting and artificial artificial neural networks (ANNs). Since the main interest of this study revolves around ANNs, this will be the focus of following chapters.

\subsubsection{Artificial Neural Networks}

Artificial neural networks are loosely inspired by neurobiological concepts of the human brain. However, they are not models of the human brain. There is no evidence that the brain implements learning like the mechanisms used in ANNs\cite{Chollet2017}.

\subsubsection{Gradient Descent} 

\subsubsection{Regression, Classification and Segmentation*}

Neural networks can help to solve different types of supervised machine learning problems of which regression, classification and segmentation are the most common.

A regression describes the prediction of one or multiple continuous outputs. An example for this would be the age prediction of a person based on their knee MRI. Regression is also a big part of object detections where it's applied to determine coordinates in an image.

A classification sorts the input into one or multiple categories, like predicting if a knee MRI is open, partially closed or closed. It can be seen as n parallel regressions where n equals the total number of classes. The continuous output for each class is the probability of the input belonging into this class. For 1 out of n classifications, the most probable category is predicted. For m out of n classifications, a threshold defines at which point a class is predicted.

A segmentation creates an image of identical dimensions as the input. Every channel of the output mask belongs to a specific category that is segmented. In this study the Femur, Tibia and Fibula needed to be masked from the rest of the MRI content. Segmenting these three on a single channel solves a different problem. Instead of distinguishing between separate classes, any type of bone will by masked as a result.

\newpage
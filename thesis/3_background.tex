\section{Background}

\subsection{Medicine}

write something

\subsubsection{Magnetic Resonance Imaging}

text

\subsubsection{Epiphyseal Plates}

text

\subsection{Computer Vision}

\subsubsection{Convolutions}

\subsection{Artificial Neural Networks}

Artificial Neural Network are based on 

\subsubsection{Convolutions}

The introduction of convolutional layers in deep learning 

\subsubsection{Regression, Classification and Segmentation*}

Neural networks can help to solve different types of supervised machine learning problems of which regression, classification and segmentation are the most common.

A regression describes the prediction of one or multiple continuous outputs. An example for this would be the age prediction of a person based on their knee MRI. Regression is also a big part of object detections where its applied to determine coordinates in an image.

A classification sorts the input into one or multiple categories, like predicting if a knee MRI is open, partially closed or closed. It can be seen as n parallel regressions where n equals the total number of classes. The continuous output for each class is the probability of the input belonging into this class. For 1 out of n classifications, the most probable category is predicted. For m out of n classifications, a threshold defines at which point a class is predicted.

A segmentation creates an image of identical dimensions as the input. Every channel of the output mask belongs to a specific category that needs to be segmented. In this study the Femur, Tibia and Fibula needed to be masked from the rest of the MRI content. Segmenting these three on a single channel solves a different problem. Instead of distinguishing between separate classes, any type of bone will by masked as a result.

\newpage
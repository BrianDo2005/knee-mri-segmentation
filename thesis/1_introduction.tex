\section{Introduction}

Recent advances in artificial intelligence have led to fully automated workflows that often exceed human performance. State of the art neural networks can classify images into thousands of categories more accurate and magnitudes faster than we can \cite{He2015a}. They translate text between multiple languages \cite{Wu2016}, drive cars autonomously through cities \cite{Bojarski2017} and detect malware in computer systems \cite{Saxe2015}. In most of these cases, they have been trained on tens of thousands or even millions of data samples. Neural networks have also found great success in the field of medical image analysis, where data sets are often much smaller. Although the same techniques can be applied, one is often confronted with a different set of challenges.

\subsection{Field and Context}

One of the main topics of forensic science is the age assessment, which is a difficult procedure to determine the chronological age of a person lacking legal documents. It plays an essential role in asylum and criminal proceedings to protect children and afford them with provisions they are entitled to by law. Due to the european migration crisis, this topic has gained a lot of public attention. There is no method that offers an exact identification of the age, but several assessments are currently used in the EU that can be separated into two groups \cite{EuropeanAsylumSupportOffice2013}.

Personal interviews or examinations are held to gain an observation of physical or psychological features. These practices include a high amount of manual work, and they are also subjective to the person conducting the test. The other type of assessments uses X-rays to observe physical traits like the collar bone, teeth or carpal growth plates \cite{EuropeanAsylumSupportOffice2013}. During the X-ray recordings the patient is exposed to ionizing radiation, which can increase the risk of long term effects like cancer \cite{WorldHealthOrganization2016}. It is an invasive imaging methods that can only be performed as part of a judicial order. For this reason, there is a high demand for a fully automated, unbiased and non-invasive method for accurately determining the age of a person.

\subsection{Research Problem}

The above requirements have led to studies evaluating the age assessment based on MRI recordings that are non-invasive. Previous publications \cite{Saring2014}\cite{Jopp2007} show that the closing process of growth plates in the knee aligns with the coming of age of teenagers. The status of the growth plates could therefore be an appropriate indicator for the age around the border to adulthood.

However, these studies are often based on qualitative comparisons, with computer-based methods being a more desirable approach. This could further reduce prejudiced results, while speeding up the process and creating a standardized measure. 

In a recent study funded by the German Research Foundation (DFG) new MRI data is collected prospectively in order to scientifically verify this hypothesis. Due to the underlying technology used in MRI machines, these images show high detail in non-bone tissue. Despite this being an advantage in many other medical applications, it adds complex information outside the scope of the bone and growth plates.

Neural networks are known to be feature selectors, meaning that they will learn to extract information that is relevant to the task \cite{Setiono1997}. This assumes that the size of the data set and the complexity of the problem enable the network to find correlating features. Based on a series of tests, I was not able to create a stable algorithm that would predict the age of a candidate based on their knee MRI. A logical next step towards this goal was the bone segmentation in MR images to reduce the complexity of the data samples.

\subsection{Previous Research}

A study led by Dodin et al. in 2011 focussed on the same goal by masking the Femur and Tibia bone from other soft tissue. They used the ray casting technique, which disassembles the MR images into several surface layers to find the boundaries of the bones. Afterward, multiple partial maps were merged for the final result \cite{Dodin2011}.

Dam et al. presented another approach in 2015 focusing on the segmentation of cartilages in the knee for the research on osteoarthritis. Their method combines a multi-atlas rigid registration and voxel classification. Besides masking medial and lateral cartilages, they also applied their technique on the Tibia \cite{Dam}.

\subsection{Focus of this Thesis}

The focus of this thesis revolves around creating a fully automated workflow that segments the long bones in 3D MRIs of human knees. Convolutional neural networks will be the base tool for this study because they have been setting state of the art results in a vast majority of image segmentation tasks. In the end, I will briefly present a proof of concept by using the segmented data to assess the age of candidates.

\newpage

\begin{abstract}

Password attacks are at the edge of accessing someones secrets. By learning to judge the strength of a password and by understanding how hackers execute attacks, users can make better estimations on how safe they are.

The entropy is widely used to measure how safe a password is, but many sources draw inaccurate conclusions between the entropy of a random password and the strength of a password that was chosen by a person. It is important to understand how these two differ and why realistic password strength is often hard to determine.

Todays hardware gives hackers incredibly powerful machines to launch different types of password attacks. Common password patterns lower possible permutations by such a magnitude that even seemingly safe passwords can be successfully attacked. In combination with frequently used passwords and personal information, hackers can further increase the effectiveness of their attacks.

By explaining common terminologies and analysing different datasets we will look at password attacks from the perspective of users, system administrators and hackers. All three benefit by understanding how the others operate in practice.

\end{abstract}
\newpage